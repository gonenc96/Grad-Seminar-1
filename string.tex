\documentclass{beamer}
\usetheme{CambridgeUS}
\usepackage[utf8]{inputenc}
\usepackage[english]{babel}
\usepackage[T1]{fontenc}
\usepackage{amsmath}
\usepackage{physics}
\usepackage{amsfonts}
\usepackage{amssymb}
\usepackage{enumerate}
% \usepackage{geometry}
\usepackage{color}
\usepackage{todonotes}
\usepackage{subcaption}
\usepackage{hyperref}
\hypersetup{
  colorlinks,
  citecolor=black,
  filecolor=black,
  linkcolor=black,
  urlcolor=black
}
\definecolor{darkred}{rgb}{0.8,0,0}
\captionsetup[subfigure]{labelformat=empty}
\usepackage{appendixnumberbeamer}
\usepackage{soul}


\usefonttheme[onlymath]{serif}
\title{Introduction to String Theory}
\author{G\"onen\c c Mo\u gol}
\date{2018--10--01}


\newcommand{\rr}{\ensuremath{\mathbb R}}
\renewcommand{\d}{\, \mathrm d}
\renewcommand{\abs}[1]{\ensuremath{\lvert #1 \rvert}}
\renewcommand{\L}{\ensuremath{\mathcal{L}}}
\let\temp\epsilon
\let\epsilon\varepsilon
\let\varepsilon\temp
\newcommand{\pt}{\ensuremath{\mathcal{P}^{\tau}_{\mu}}}
\newcommand{\ps}{\ensuremath{\mathcal{P}^{\sigma}_{\mu}}}
\newcommand{\ptt}{\ensuremath{\mathcal{P}\indices{^{\tau}^{\mu}}}}
\newcommand{\pss}{\ensuremath{\mathcal{P}\indices{^{\sigma}^{\mu}}}}
\newcommand{\vperp}{\ensuremath{ v_{\perp}}}
\renewcommand{\tilde}{\widetilde}
\renewcommand{\qc}{\quad ,}

\newcommand{\red}[1]{{\color{red} #1 }}
\newcommand{\onlyi}[2]{\only<#1>{\item #2}}
\newcommand{\zz}{\mathbb Z}


\begin{document}

\begin{frame}
  \maketitle
\end{frame}

\title{Introduction to String Theor\textbf{ies}}

\begin{frame}
  \maketitle
\end{frame}

\title{Introduction to String Theor{ies}}

\begin{frame}
  \frametitle{Overview}
  \tableofcontents{}
\end{frame}

\section{Motivation}



\begin{frame}
  \frametitle{Why String Theory?}
  \begin{itemize}
  \uncover<1>{\item {It is fun}}
  \item<2-> \st{It is fun} \pause
  \item UV-finite
  \item Provides a theory of quantum gravity
  \item Creates tools for other areas of physics and mathematics
  \item Has the right ingredients for producing standard model

  \end{itemize}
\end{frame}

\begin{frame}
  \frametitle{Bosonic String}
  

  \begin{itemize}
    \item<1-> {Units and conventions: $c=1$, $\hbar = 1$, $\eta = \mathrm{diag}(-1,+1,\dots, +1)$ }
    \onlyi 2 { Relativistic particle Lagrangian:
      \[ S = - m \int {\d \tau} \]}

    \onlyi 3 {Relativistic particle Lagrangian:
      \[ S = - m \int \red{\d \tau} \]}
    \onlyi 4 {Relativistic particle Lagrangian:
      \[ S = - \red{m} \int \d \tau \] }
  \item<5-> Relativistic particle Lagrangian:
    \[ S = - m \int {\d \tau} \]
    \onlyi 5{ Generalize to an extended object:
      \[ S = -T \int \d \mathcal{A}  \]}
    \onlyi 6 {Generalize to an extended object:
      \[ S = -T \int \red{\d \mathcal{A}}  \]}
    \onlyi 7 {Generalize to an extended object:
      \[ S = -\red T \int \d \mathcal{A}  \]}
%  \item Add picture
  \end{itemize}
  
\end{frame}

\section{Bosonic String}




\begin{frame}
  \frametitle{Bosonic String}
  \begin{itemize}
  \item<1-> Manipulations of equation of motion result in wave equation
    \item<2-> { Left and right moving weaves $X^{\mu} = 1/2\, (X^{\mu}_{L}+ X^{\mu}_{R})$}
      \onlyi 3 {With Fourier expansions:
        \[ X^{\mu}_{L}(\tau + \sigma) = {x^{\mu}} + \frac{ {p^{\mu}} l^{2}}{2} \cdot (\tau + \sigma) + i l \sum_{n \neq 0 } \frac 1n  \, {\alpha^{\mu}_{n}} \,  e^{-in(\tau+ \sigma)}  \]
    \[ X^{\mu}_{R}(\tau - \sigma) = x^{\mu} + \frac{p^{\mu} l^{2}}{2} \cdot (\tau - \sigma) + i l \sum_{n \neq 0 } \frac 1n\, \tilde \alpha^{\mu}_{n} \, e^{-in(\tau- \sigma)}  \] }
  \onlyi 4 {With Fourier expansions:
    \[ X^{\mu}_{L}(\tau + \sigma) = {x^{\mu}} + \frac{ {p^{\mu}} \red{l^{2}}}{2} \cdot (\tau + \sigma) + i \red{l} \sum_{n \neq 0 } \frac 1n  \, {\alpha^{\mu}_{n}} \,  e^{-in(\tau+ \sigma)}  \]
    \[ X^{\mu}_{R}(\tau - \sigma) = x^{\mu} + \frac{p^{\mu} \red{l^{2}}}{2} \cdot (\tau - \sigma) + i \red{l} \sum_{n \neq 0 } \frac 1n\, \tilde \alpha^{\mu}_{n} \, e^{-in(\tau- \sigma)}  \] }
  \onlyi 5 { With Fourier expansions:
    \[ X^{\mu}_{L}(\tau + \sigma) = \red{x^{\mu}} + \frac{ \red{p^{\mu}} l^{2}}{2} \cdot (\tau + \sigma) + i l \sum_{n \neq 0 } \frac 1n  \, \red{\alpha^{\mu}_{n}} \,  e^{-in(\tau+ \sigma)}  \]
    \[ X^{\mu}_{R}(\tau - \sigma) = \red{x^{\mu}} + \frac{\red{p^{\mu} }l^{2}}{2} \cdot (\tau - \sigma) + i l \sum_{n \neq 0 } \frac 1n\,\red{ \tilde \alpha^{\mu}_{n}} \, e^{-in(\tau- \sigma)}  \] }


\item<6-> { With Fourier expansions:
    \[ X^{\mu}_{L}(\tau + \sigma) = {x^{\mu}} + \frac{ {p^{\mu}} l^{2}}{2} \cdot (\tau + \sigma) + i l \sum_{n \neq 0 } \frac 1n  \, {\alpha^{\mu}_{n}} \,  e^{-in(\tau+ \sigma)}  \]
    \[ X^{\mu}_{R}(\tau - \sigma) = x^{\mu} + \frac{p^{\mu} l^{2}}{2} \cdot (\tau - \sigma) + i l \sum_{n \neq 0 } \frac 1n\, \tilde \alpha^{\mu}_{n} \, e^{-in(\tau- \sigma)}  \] }

  
\item <6-> For the closed string, $\alpha^{\mu}_{n}$ and $\tilde \alpha^{\mu}_{n}$ are independent. 
  \end{itemize}
\end{frame}

\subsection{Quantisation}




\begin{frame}
  \frametitle{Quantisation}
  \begin{itemize}
  \item Promote Fourier modes to operators with commutation relations
    \[ [x^{\mu}, p_{\nu} ] = i \delta^{\mu}_{\nu} \qquad [\alpha^{\mu}_{n}, \alpha^{\nu}_{m}] = n \delta_{n+m} \eta^{\mu\nu} = [\tilde \alpha^{\mu}_{n}, \tilde \alpha^{\nu}_{m}]  \] \pause
  \item Define vacuum $\ket 0$ by imposing
    \[ \alpha_{n}^{\mu} \ket 0 = \tilde \alpha_{n}^{\mu} \ket 0 = 0 \quad \text{for all } n>0 \]\pause
  \item For $n>0$,  $\alpha_{n}^{\mu}$ and $\tilde \alpha^{\mu}_{n}$ are annihilation operators
  \item For $n<0 $,   $\alpha^{\mu}_{n}$ and $\tilde \alpha^{\mu}_{n}$ are creation operators
  \end{itemize}
\end{frame}

\begin{frame}
\frametitle{Pretty Pictures of String Excitations}

% \begin{figure}[h]
%   \centering
  
%   \begin{subfigure}{0.3\textwidth}
%     \includegraphics[width=\textwidth]{first.png}
%     \caption{$\ket 0$}
%   \end{subfigure}  \pause
%   \begin{subfigure}{0.3\textwidth}
%     \includegraphics[width=\textwidth]{second.png}
%     \caption{\\{$\alpha^{\mu}_{-1} \tilde\alpha^{\nu}_{-1} \ket 0$}}
%   \end{subfigure} 
%   \begin{subfigure}{0.3\textwidth}
%     \includegraphics[width=\textwidth]{third.png}
%     \caption{$\alpha^{\mu}_{-2} \tilde \alpha^{\nu}_{-2} \ket 0$}
%   \end{subfigure}

% \end{figure}

\end{frame}

\subsection{Closed Spectrum}

\begin{frame}
\frametitle{Closed Spectrum}
\begin{itemize}
\item There is a level matching condition $N= \tilde N$
\item Vacuum is a tachyon, with mass $M^{2}<0$ !
\item First excited state is $\ket{\boldsymbol {\xi}} =  \xi_{ij} \alpha_{-1}^{i} \tilde \alpha^{j}_{-1} \ket 0  $ with $M^{2}=0$
\item Decompose $\xi_{ij}$ in irreps
  \[\xi_{ij} = \underbrace{\xi_{(ij)}}_{\text{symmetric}} +\underbrace{ \xi_{[ij]}}_{\text{antisymmetric}} +\underbrace{ \xi^{(0)} \delta_{ij}}_{\text{trace part}}\]
  
\end{itemize}
\end{frame}

\begin{frame}
  \frametitle{Closed Spectrum}
  \begin{itemize}
  \item Decompose $\xi_{ij}$ in irreps
  \[\xi_{ij} = \underbrace{\xi_{(ij)}}_{\text{symmetric}} +\underbrace{ \xi_{[ij]}}_{\text{antisymmetric}} +\underbrace{ \xi^{(0)} \delta_{ij}}_{\text{trace part}}\]
  \item Symmetric part describes spin 2 gauge field $\implies$ graviton!
  \item Antisymmetric part is called \emph{Kalb-Ramon Field}.
  \item Trace part describes the \emph{dilaton}.
  \end{itemize}
\end{frame}

\section{Superstring}



\begin{frame}
  \frametitle{Superstring}
  \begin{itemize}
  \item Modify the Lagrangian to include fermionic fields
    \[ S = S_{\text{boson}} + S_{\text{fermion}}\]
  \item By supersymmetry, fermions are related to bosons. \pause
  \item Supersymmetry therefore forbids a tachyonic state. \pause
  \item For left and right moving parts, we can choose either spin 1 bosons ``B''
  \item Or spin 1/2 fermions ``F$_{\pm}$'' with two chiralities

  \end{itemize}
  
\end{frame}




\begin{frame}
  \frametitle{Decomposition into Irreps}
  \begin{itemize}
  \item (B,B) = Dilaton $\oplus$ 2-Form Field (Kalb-Ramon) $\oplus$ Graviton
  \item (F$_\pm$, F$_{\pm}$) = certain $k$-form Fields
  \item (B,F$_{+}$) = Dilatino $\oplus$ Gravitino
  \item (B,F$_{-}$) = Dilatino $\oplus$ Gravitino (with different chirality)
  \end{itemize}
\end{frame}

\subsection{Type IIA/B}



\begin{frame}
  \frametitle{Type IIA/B}
  \begin{itemize}
  \item Consistency conditions limit the sectors that one can choose. \pause
  \item Type IIB consists of
    \begin{itemize}
    \item (B,B): Dilaton, Kalb-Ramon, Graviton
    \item (F$_{+}$,F$_{+}$): $k$-form Fields
    \item (B,F$_{+}$): Dilatino, Gravitino
    \item (F$_{+}$,B): Dilatino, Gravitino
    \end{itemize}\pause
    
  \item Type IIA consists of
    \begin{itemize}
    \item (B,B): Dilaton, Kalb-Ramon, Graviton
    \item (F$_{+}$,F$_{-}$): $k$-form Fields
    \item (B,F$_{-}$): Dilatino, Gravitino (different chirality)
    \item (F$_{+}$,B): Dilatino, Gravitino
    \end{itemize}

  \end{itemize}
\end{frame}

\subsection{Type I}



\begin{frame}
  \frametitle{Type I}
  \begin{itemize}
  \item Type IIB consists of
    \begin{itemize}
    \item (B,B): Dilaton, Kalb-Ramon, Graviton
    \item (F$_{+}$,F$_{+}$): $k$-form Fields
    \item (B,F$_{+}$): Dilatino, Gravitino
    \item (F$_{+}$,B): Dilatino, Gravitino
    \end{itemize}  \item Type IIB theory is invariant under exchanging left and right movers. \pause
  \item We call this symmetry $\Omega$, then \emph{closed spectrum} of type I theory is $\text{IIB}/\Omega$ i.e. it consists of
    \begin{itemize}
    \item Dilaton, Graviton
    \item Some of the $k$-form fields
    \item Dilatino, Gravitino
    \end{itemize}
\item The closed spectrum of type I theory is inconsistent by itself
\item Add open strings with gauge group $SO(32)$ to complete type I theory.
\item Type I theory is in some sense ``half'' of Type IIB theory.
\end{itemize}
  
  
  
\end{frame}

\subsection{Heterotic Strings}



\begin{frame}
  \frametitle{Heterotic Strings}
  \begin{itemize}
  \item There are yet another consistent theories, which takes bosonic strings on left moving and supersymmetric strings on right moving sector.
  \item Bosonic part is compactified
  \item This gives two new theories with gauge groups $SO(32)$ and $E_{8} \times E_{8}$
  \end{itemize}
  
\end{frame}

\begin{frame}
      \frametitle{Bibliography}
    \begin{thebibliography}{}
    \bibitem{article}
      Weigand, Timo
      \textit{Introduction to String Theory, Lecture Notes}\\
    \end{thebibliography}
\end{frame}


\appendix

\begin{frame}
  \frametitle{Ramon and Neveu-Schwarz Sectors}
  \begin{itemize}
  \item If $\psi_{L,R} (\sigma) = + \psi_{L,R}(\sigma+ 2\pi)$ then $\psi$ is in \emph{Ramon sector} with \alert{integer} Fourier expansion
    \[ \psi_{R}^{\mu} = \sum_{n\in \zz}  b_{n}^{\mu} e^{-in(\tau-\sigma)}   
      \qquad \psi_{L}^{\mu} = \sum_{n\in \zz}  \tilde b_{n}^{\mu} e^{-in(\tau+\sigma)}   \]
  \item If $\psi_{L,R} (\sigma) = - \psi_{L,R}(\sigma+ 2\pi)$ then $\psi$ is in \emph{Neveu-Schwarz sector} with \alert{half integer} Fourier expansion
    \[ \psi_{R}^{\mu} = \sum_{r \in \zz+ \frac 12}  b_{r}^{\mu} e^{-ir(\tau-\sigma)}   
      \qquad \psi_{L}^{\mu} = \sum_{r \in \zz+\frac 12}  \tilde b_{r}^{\mu} e^{-ir(\tau+\sigma)}   \]
    
  \item So there are total of 4 choices possible:$\text{ (NS,NS) (NS,R) (R,NS) (R,R)}$
  \end{itemize}
\end{frame}

\begin{frame}
  \frametitle{Quantisation}
  \begin{itemize}
  \item Promote Fourier modes to operators with anti-commutation relations
    \[ \{b^{\mu}_{n}, b^{\nu}_{m}\} = \eta^{\mu\nu} \delta_{n+m} =  \{ \tilde b^{\mu}_{n}, \tilde b^{\nu}_{m}\}  \]
  \item Define Vacua:
    \[ \alpha_{n}^{\mu} \ket 0_{NS} = b^{\mu}_{r} \ket 0_{NS} = \tilde b^{\mu}_{r} \ket 0_{NS} = 0 \quad \forall n,r>0  \]
    \[ \alpha_{n}^{\mu} \ket 0_{R} = b_{m}^{\mu} \ket 0_{R} = \tilde b^{\mu}_{m} \ket 0_{NS} = 0 \quad \forall n,m >0    \]

  \item \alert{Improtant:} NS ground state is well defined and is a space-time scalar. However, R ground state is degenerate as
    \[ b^{\mu}_{m} b^{\nu}_{0} \ket 0 _{R} = - b^{\nu}_{0} b^{\mu}_{m} \ket 0_{R} = 0  \]
  \item Because of this degeneracy, one can show that $\ket 0_{R}$ is a fermion.
    
  \end{itemize}
\end{frame}

\begin{frame}
  \frametitle{Massless Spectrum}
  \begin{itemize}
  \item Recall that $\ket 0_{R}$ is a fermion and $\ket 0_{NS}$ is a scalar.
  \end{itemize}
  \center{    \renewcommand*{\arraystretch}{1.4}
  \begin{tabular}{c|c|c}
    Sector (L,R) & State & $SO(8) $ rep \\
    \hline 
    (NS,NS) & $b^{i}_{-1/2} \ket 0_{NS} \otimes \tilde b^{i}_{-1/2} \ket 0_{NS}$ & $8_{v} \otimes 8_{v}$ \pause \\
                 (R$_{+}$,R$_{+}$)& $\ket +_{R} \otimes \ket +_{R}$ & $8_{s} \otimes 8_{s}$  \\
    (R$_{+}$,R$_{-}$)& $\ket +_{R} \otimes \ket -_{R}$ & $8_{s} \otimes 8_{c}$ \\
    (R$_{-}$,R$_{-}$)& $\ket -_{R} \otimes \ket -_{R}$& $8_{c} \otimes 8_{c}$ \pause \\
    (NS,R$_{+}$) & $b^{i}_{-1/2} \ket 0 _{NS} \otimes \ket + _{R}$ & $8_{v} \otimes 8_{s}$ \\
    (NS,R$_{-}$) & $b^{i}_{-1/2} \ket 0 _{NS} \otimes \ket - _{R}$ & $8_{v} \otimes 8_{c}$
  \end{tabular} }
\end{frame}




\end{document}

%%% Local Variables:
%%% mode: latex
%%% TeX-master: t
%%% End:
